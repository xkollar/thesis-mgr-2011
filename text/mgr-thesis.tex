% vim: spell
%\documentclass[11pt,oneside,final]{fithesis2}
\documentclass[11pt,oneside,draft]{fithesis2}

%\usepackage[draft]{pgf}
\usepackage{tikz}

%\usepackage[slovak]{babel}
\usepackage[english]{babel}
\usepackage[utf8]{inputenc}
\usepackage[T1]{fontenc}
\usepackage{caption}

\usepackage{lmodern}
\usepackage{chemarrow}
\usepackage{amsmath,amsthm,amsfonts}
\usepackage{mathrsfs}
%\usepackage{ae}
\usepackage{fancyvrb}
\usepackage{enumerate}
\usepackage{mathpartir}
\usepackage{comment}

%\usepackage[nottoc]{tocbibind}
%\usepackage{tocbibind}

%\usepackage[plainpages=false,pdfpagelabels,unicode,colorlinks]{hyperref}
\usepackage[plainpages=false,pdfpagelabels,unicode]{hyperref}
\usepackage[all]{hypcap}

\clubpenalty=1000
\widowpenalty=1000
\renewcommand{\baselinestretch}{1.08}

\newcommand\uv[1]{``#1''}
\renewcommand{\to}{\mathrel{\mathop{\chemarrow}}}
%\let\to\chemarrow

%% Symbols used in generated graphics
%\newcommand{\kindStar}{\star}
\newcommand{\kindStar}{{\star}}
\newcommand{\hasType}{~{:}~}
\newcommand{\hasKind}{\,{:}\,}
\newcommand{\typeTo}{\mathrel{\mathop{\to}}}
\newcommand{\kindTo}{\mathrel{\mathop{\to}}}
\newcommand{\typeDot}{\,.\,}
\newcommand{\inj}{\hookrightarrow}


\DefineVerbatimEnvironment{code}{Verbatim}
        { fontsize=\small
        , commentchar=@
        , frame=single
        %, numbers=left
        %, framesep=5pt
        %, fontshape=sl
        , commandchars=\\\{\}
        , framesep=5pt
        %, fontshape=sl
        }

\newtheorem{lemma}{Lemma}

\theoremstyle{definition}
\newtheorem{example}{Example}

%\thesistitle{Rozšírenia typového systému pre Haskell}
\thesistitle{Extensions of Haskell type system}
%\thesissubtitle{Diplomová práca}
%\thesissubtitle{Master Thesis}
\thesissubtitle{Diploma Thesis}
\thesisstudent{Matej Kollár}
\thesiswoman{false}
\thesisfaculty{fi}
%\thesisyear{jar 2011}
\thesisyear{autumn 2011}
\thesisadvisor{Libor Škarvada, RNDr.}
%\thesislang{sk}
\thesislang{en}

\begin{document}
\FrontMatter
\ThesisTitlePage

\begin{ThesisDeclaration}
\addcontentsline{toc}{chapter}{Declaration}
\DeclarationText
\AdvisorName
\end{ThesisDeclaration}

\begin{ThesisThanks}
\addcontentsline{toc}{chapter}{Acknowledgement}
I would like to thank to Libor, Zuzana and mommy.
\end{ThesisThanks}

\begin{ThesisAbstract}
\addcontentsline{toc}{chapter}{Abstract}
The aim of the master's thesis/work is to provide...
\end{ThesisAbstract}

% XXX! Check whether they are relevant.
\begin{ThesisKeyWords}
\addcontentsline{toc}{chapter}{Keywords}
Haskell, type system, type extension
\end{ThesisKeyWords}

\clearpage
\phantomsection
\addcontentsline{toc}{chapter}{\contentsname}
\tableofcontents

\clearpage
\phantomsection
\addcontentsline{toc}{chapter}{\listfigurename}
\listoffigures

\MainMatter

\chapter{Introduction}

\chapter{Type system}

% \begin{align*}
% 	\mathscr{A} & \to \mathscr{BCDEFGHIJKLMNOPQRSTUVWXYZ} \\
% 	\mathcal{A} & \to \mathcal{BCDEFGHIJKLMNOPQRSTUVWXYZ}
% \end{align*}

% XXX: Move this to introduction?
Talking about extensions of type system requires understanding what type system is.
In this work we will understand type system (similarly as
\cite{pierce:2002:types,barendregt:1992:lambdaProc}) a formal system providing

\begin{description}
	\item[safety] -- meaningless or invalid code can be detected at compile time,
	\item[optimization] -- useful information might be obtained as a by-product of static type-checking,
	\item[documentation] -- types provide information on how various parts can be combined together,
	\item[abstraction] -- types provide higher level of thinking about programs,
	\item[specification] -- specifying type narrows possibilities how to write down programs,
		eliminating many erroneous possibilities.
\end{description}

These qualities are typically achieved by meta system, typically
enriching syntax of a language and set of rules that restrict how can
simpler programs combined into more complex ones.
Check for these rules is done statically (at compile time).
%Static typing, Strong typing, \dots

\section{Haskell type system}

% It is desirable that we give description of the original type system before
% discussing its extensions.
Before discussing extensions it would be desirable to give description
of the original type system.
As we can find not only in \cite{haskell2010}, Haskell uses a traditional
Hindley-Milner polymorphic type system to provide a static type
semantics enriched with type classes that provide structural way to
introduce overloaded functions.

Types and terms are two different things, types are polymorphic,
type variables are implicitly universally quantified at the top level of type.
Type system provides not only type checking but also type inference.

Basic/Constant types
\begin{code}
()   :: ()
True :: Bool
'a'  :: Char
1    :: Integer
\end{code}

Type Constructors
\begin{code}
[1, 2, 3]        :: [Integer]
(True, 'a')      :: (Bool, Char)
Just [((), "a")] :: Maybe [((), String)]
isAlpha          :: Char -> Bool
\end{code}

Polymorphic types
\begin{code}
id    :: a -> a
const :: a -> b -> a
\end{code}

What do these types mean? Type of identity function tells us, that
it can take as argument value of any type and it returns value of the
same type. Well, not exactly any type. We are interested in inhabited
types only. But we know \uv{category} that consists only of these types~--
kind \(\kindStar\). Let us write that in explicit way.

%\begin{example}~
\begin{align*}
\texttt{id} & :: \forall a \in {\kindStar} \, . \, a \to a \\
\texttt{const} & :: \forall a \in {\kindStar} \, . \, \forall b \in {\kindStar} \, . \, a \to b \to a \\
\end{align*}
% \[ \texttt{id} :: \forall a \in {\kindStar} \, . \, a \to a \]
% \[ \texttt{const} :: \forall a \in {\kindStar} \, . \, \forall b \in {\kindStar} \, . \, a \to b \to a \]
%\end{example}

Somewhere between concrete and polymorphic lies values with polymorphic overloading.
\begin{code}
1    :: Num a => a
elem :: Eq a => a -> [a] -> Bool
sum  :: Num a => [a] -> a
\end{code}
What do these types mean? There is clearly some sort of quantification, but not over all
possible \(\kindStar\) types. Those classes can therefore be taken as subkinds of \(\kindStar\).
Again, let us make this notion more explicit.

%\begin{example}~
\begin{align*}
\texttt{1} & :: \forall a \in \texttt {Num} \, . \, a \\
\texttt{elem} & :: \forall a \in \texttt{Eq} \, . \, a \to [a] \to \texttt{Bool} \\
\texttt{sum} & :: \forall a \in \texttt{Num} \, . \, [a] \to a \\
% \[ \texttt{1} :: \forall a \in \texttt {Num} \, . \, a \]
% \[ \texttt{elem} :: \forall a \in \texttt{Eq} \, . \, a \to [a] \to \texttt{Bool} \]
% \[ \texttt{sum} :: \forall a \in \texttt{Num} \, . \, [a] \to a \]
\end{align*}
%\end{example}

And we somehow know, that \( \texttt{Eq}, \texttt{Num} \subseteq \kindStar \).
Standard Prelude \(\kindStar\)-classes and subkind relation on them is
captured in figure~\ref{diagram:kinds_star1}.
\noindent
\begin{figure}
	\centering
	\scalebox{0.9}{
\begin{tikzpicture}[>=latex,line join=bevel,]
  \pgfsetlinewidth{1bp}
%%
\pgfsetcolor{black}
  % Edge: \texttt{Eq} -> \texttt{Num}
  \draw [->] (167.61bp,204.36bp) .. controls (172.91bp,200.42bp) and (178.92bp,195.93bp)  .. (192.67bp,185.69bp);
  % Edge: \texttt{Show} -> \texttt{Num}
  \draw [->] (207bp,205.23bp) .. controls (207bp,202.29bp) and (207bp,199bp)  .. (207bp,185.56bp);
  % Edge: \kindStar -> \texttt{Read}
  \pgfsetcolor{gray}
  \draw [->] (138.79bp,252.83bp) .. controls (120.95bp,248.45bp) and (88.037bp,240.01bp)  .. (42.198bp,224.81bp);
  % Edge: \kindStar -> \texttt{Enum}
  \draw [->] (165.04bp,252.16bp) .. controls (186.09bp,245.97bp) and (228.14bp,233.61bp)  .. (266.82bp,222.23bp);
  % Edge: \texttt{Ix} -> \emptyset
  \draw [->,dotted] (102.47bp,124.08bp) .. controls (101.48bp,103.92bp) and (99.364bp,60.469bp)  .. (97.575bp,23.787bp);
  % Edge: \kindStar -> \texttt{Show}
  \draw [->] (165.03bp,246.52bp) .. controls (170.71bp,242.39bp) and (177.54bp,237.42bp)  .. (192.51bp,226.54bp);
  % Edge: \texttt{Floating} -> \texttt{RealFloat}
  \pgfsetcolor{black}
  \draw [->] (157.85bp,82.361bp) .. controls (158.05bp,79.7bp) and (158.26bp,76.793bp)  .. (159.22bp,63.685bp);
  % Edge: \texttt{RealFloat} -> \emptyset
  \pgfsetcolor{gray}
  \draw [->,dotted] (143.78bp,42.441bp) .. controls (136.17bp,37.493bp) and (127bp,31.525bp)  .. (110.41bp,20.725bp);
  % Edge: \texttt{Fractional} -> \texttt{Floating}
  \pgfsetcolor{black}
  \draw [->] (169.53bp,124.23bp) .. controls (168.35bp,121.38bp) and (167.04bp,118.21bp)  .. (161.85bp,105.71bp);
  % Edge: \texttt{Num} -> \texttt{Fractional}
  \draw [->] (198.33bp,164.49bp) .. controls (195.55bp,161.12bp) and (192.38bp,157.28bp)  .. (182.73bp,145.58bp);
  % Edge: \kindStar -> \texttt{Eq}
  \pgfsetcolor{gray}
  \draw [->] (152bp,246.32bp) .. controls (152bp,243.67bp) and (152bp,240.7bp)  .. (152bp,227.59bp);
  % Edge: \texttt{Real} -> \texttt{Integral}
  \pgfsetcolor{black}
  \draw [->] (268.28bp,124.44bp) .. controls (275.57bp,120.24bp) and (284.13bp,115.29bp)  .. (301.03bp,105.53bp);
  % Edge: \texttt{Read} -> \emptyset
  \pgfsetcolor{gray}
  \draw [->,dotted] (21bp,205.26bp) .. controls (21bp,189.74bp) and (21bp,160.17bp)  .. (21bp,135bp) .. controls (21bp,135bp) and (21bp,135bp)  .. (21bp,94bp) .. controls (21bp,62.07bp) and (52.381bp,37.317bp)  .. (83.762bp,18.559bp);
  % Edge: \texttt{Enum} -> \texttt{Integral}
  \pgfsetcolor{black}
  \draw [->] (290.93bp,205.17bp) .. controls (296.36bp,185.08bp) and (308.1bp,141.69bp)  .. (317.84bp,105.69bp);
  % Edge: \texttt{Bounded} -> \emptyset
  \pgfsetcolor{gray}
  \draw [->,dotted] (82.342bp,205.46bp) .. controls (73.504bp,190.43bp) and (59bp,161.69bp)  .. (59bp,135bp) .. controls (59bp,135bp) and (59bp,135bp)  .. (59bp,94bp) .. controls (59bp,71.309bp) and (71.127bp,48.087bp)  .. (87.607bp,23.71bp);
  % Edge: \texttt{Ord} -> \texttt{Ix}
  \pgfsetcolor{black}
  \draw [->] (138.39bp,164.49bp) .. controls (133.8bp,160.66bp) and (128.48bp,156.23bp)  .. (115.69bp,145.58bp);
  % Edge: \texttt{Eq} -> \texttt{Ord}
  \draw [->] (151.72bp,204.36bp) .. controls (151.65bp,201.7bp) and (151.58bp,198.79bp)  .. (151.26bp,185.69bp);
  % Edge: \texttt{Ord} -> \texttt{Real}
  \draw [->] (170.09bp,167.29bp) .. controls (184.05bp,161.65bp) and (203.27bp,153.88bp)  .. (228.91bp,143.52bp);
  % Edge: \texttt{Num} -> \texttt{Real}
  \draw [->] (218.3bp,164.49bp) .. controls (222.22bp,160.84bp) and (226.72bp,156.65bp)  .. (238.63bp,145.58bp);
  % Edge: \texttt{Fractional} -> \texttt{RealFrac}
  \draw [->] (190.74bp,124.44bp) .. controls (197.66bp,120.08bp) and (205.84bp,114.92bp)  .. (222.09bp,104.67bp);
  % Edge: \texttt{Real} -> \texttt{RealFrac}
  \draw [->] (247.11bp,124.23bp) .. controls (246.3bp,121.19bp) and (245.38bp,117.78bp)  .. (241.83bp,104.56bp);
  % Edge: \texttt{Integral} -> \emptyset
  \pgfsetcolor{gray}
  \draw [->,dotted] (297.66bp,82.42bp) .. controls (274.33bp,71.137bp) and (237.25bp,53.979bp)  .. (204bp,42bp) .. controls (175.58bp,31.763bp) and (141.91bp,22.827bp)  .. (110.04bp,15.008bp);
  % Edge: \texttt{RealFrac} -> \texttt{RealFloat}
  \pgfsetcolor{black}
  \draw [->] (218.66bp,83.441bp) .. controls (209.72bp,78.803bp) and (199.05bp,73.268bp)  .. (180.28bp,63.523bp);
  % Edge: \kindStar -> \texttt{Bounded}
  \pgfsetcolor{gray}
  \draw [->] (138.95bp,247.71bp) .. controls (131.78bp,243.16bp) and (122.64bp,237.36bp)  .. (105.64bp,226.56bp);
  % Node: \texttt{Show}
\begin{scope}
  \definecolor{strokecol}{rgb}{0.0,0.0,0.0};
  \pgfsetstrokecolor{strokecol}
  \draw (228bp,227bp) -- (186bp,227bp) -- (186bp,206bp) -- (228bp,206bp) -- cycle;
  \draw (207bp,216bp) node {$\texttt{Show}$};
\end{scope}
  % Node: \texttt{Ord}
\begin{scope}
  \definecolor{strokecol}{rgb}{0.0,0.0,0.0};
  \pgfsetstrokecolor{strokecol}
  \draw (170bp,186bp) -- (132bp,186bp) -- (132bp,165bp) -- (170bp,165bp) -- cycle;
  \draw (151bp,175bp) node {$\texttt{Ord}$};
\end{scope}
  % Node: \texttt{Enum}
\begin{scope}
  \definecolor{strokecol}{rgb}{0.0,0.0,0.0};
  \pgfsetstrokecolor{strokecol}
  \draw (309bp,227bp) -- (267bp,227bp) -- (267bp,206bp) -- (309bp,206bp) -- cycle;
  \draw (288bp,216bp) node {$\texttt{Enum}$};
\end{scope}
  % Node: \texttt{Ix}
\begin{scope}
  \definecolor{strokecol}{rgb}{0.0,0.0,0.0};
  \pgfsetstrokecolor{strokecol}
  \draw (119bp,146bp) -- (87bp,146bp) -- (87bp,125bp) -- (119bp,125bp) -- cycle;
  \draw (103bp,135bp) node {$\texttt{Ix}$};
\end{scope}
  % Node: \texttt{Integral}
\begin{scope}
  \definecolor{strokecol}{rgb}{0.0,0.0,0.0};
  \pgfsetstrokecolor{strokecol}
  \draw (353bp,106bp) -- (289bp,106bp) -- (289bp,83bp) -- (353bp,83bp) -- cycle;
  \draw (321bp,94bp) node {$\texttt{Integral}$};
\end{scope}
  % Node: \texttt{Bounded}
\begin{scope}
  \definecolor{strokecol}{rgb}{0.0,0.0,0.0};
  \pgfsetstrokecolor{strokecol}
  \draw (118bp,227bp) -- (60bp,227bp) -- (60bp,206bp) -- (118bp,206bp) -- cycle;
  \draw (89bp,216bp) node {$\texttt{Bounded}$};
\end{scope}
  % Node: \kindStar
\begin{scope}
  \definecolor{strokecol}{rgb}{0.75,0.75,0.75};
  \pgfsetstrokecolor{strokecol}
  \draw (165bp,266bp) -- (139bp,266bp) -- (139bp,247bp) -- (165bp,247bp) -- cycle;
  \definecolor{strokecol}{rgb}{0.0,0.0,0.0};
  \pgfsetstrokecolor{strokecol}
  \draw (152bp,256bp) node {$\kindStar$};
\end{scope}
  % Node: \texttt{RealFloat}
\begin{scope}
  \definecolor{strokecol}{rgb}{0.0,0.0,0.0};
  \pgfsetstrokecolor{strokecol}
  \draw (195bp,64bp) -- (125bp,64bp) -- (125bp,43bp) -- (195bp,43bp) -- cycle;
  \draw (160bp,53bp) node {$\texttt{RealFloat}$};
\end{scope}
  % Node: \texttt{RealFrac}
\begin{scope}
  \definecolor{strokecol}{rgb}{0.0,0.0,0.0};
  \pgfsetstrokecolor{strokecol}
  \draw (271bp,105bp) -- (207bp,105bp) -- (207bp,84bp) -- (271bp,84bp) -- cycle;
  \draw (239bp,94bp) node {$\texttt{RealFrac}$};
\end{scope}
  % Node: \texttt{Fractional}
\begin{scope}
  \definecolor{strokecol}{rgb}{0.0,0.0,0.0};
  \pgfsetstrokecolor{strokecol}
  \draw (211bp,146bp) -- (137bp,146bp) -- (137bp,125bp) -- (211bp,125bp) -- cycle;
  \draw (174bp,135bp) node {$\texttt{Fractional}$};
\end{scope}
  % Node: \texttt{Floating}
\begin{scope}
  \definecolor{strokecol}{rgb}{0.0,0.0,0.0};
  \pgfsetstrokecolor{strokecol}
  \draw (189bp,106bp) -- (125bp,106bp) -- (125bp,83bp) -- (189bp,83bp) -- cycle;
  \draw (157bp,94bp) node {$\texttt{Floating}$};
\end{scope}
  % Node: \texttt{Real}
\begin{scope}
  \definecolor{strokecol}{rgb}{0.0,0.0,0.0};
  \pgfsetstrokecolor{strokecol}
  \draw (271bp,146bp) -- (229bp,146bp) -- (229bp,125bp) -- (271bp,125bp) -- cycle;
  \draw (250bp,135bp) node {$\texttt{Real}$};
\end{scope}
  % Node: \texttt{Num}
\begin{scope}
  \definecolor{strokecol}{rgb}{0.0,0.0,0.0};
  \pgfsetstrokecolor{strokecol}
  \draw (226bp,186bp) -- (188bp,186bp) -- (188bp,165bp) -- (226bp,165bp) -- cycle;
  \draw (207bp,175bp) node {$\texttt{Num}$};
\end{scope}
  % Node: \emptyset
\begin{scope}
  \definecolor{strokecol}{rgb}{0.75,0.75,0.75};
  \pgfsetstrokecolor{strokecol}
  \draw (110bp,24bp) -- (84bp,24bp) -- (84bp,1bp) -- (110bp,1bp) -- cycle;
  \definecolor{strokecol}{rgb}{0.0,0.0,0.0};
  \pgfsetstrokecolor{strokecol}
  \draw (97bp,12bp) node {$\emptyset$};
\end{scope}
  % Node: \texttt{Eq}
\begin{scope}
  \definecolor{strokecol}{rgb}{0.0,0.0,0.0};
  \pgfsetstrokecolor{strokecol}
  \draw (168bp,228bp) -- (136bp,228bp) -- (136bp,205bp) -- (168bp,205bp) -- cycle;
  \draw (152bp,216bp) node {$\texttt{Eq}$};
\end{scope}
  % Node: \texttt{Read}
\begin{scope}
  \definecolor{strokecol}{rgb}{0.0,0.0,0.0};
  \pgfsetstrokecolor{strokecol}
  \draw (42bp,227bp) -- (0bp,227bp) -- (0bp,206bp) -- (42bp,206bp) -- cycle;
  \draw (21bp,216bp) node {$\texttt{Read}$};
\end{scope}
%
\end{tikzpicture}

}
	\caption[Explicit \uv{star} Prelude classes]{Explicit \(\kindStar\) Prelude classes}
	\label{diagram:kinds_star1}
\end{figure}
One might have doubts about including \(\emptyset\) (with evident semantics) into diagram.
But the notion is natural (class with no instances) and it will become useful in a moment.

There are not uncommon cases where there are more than one \uv{restriction} on type variable.
Consider following example.
% \begin{example}~
\begin{code}
ex :: (C1 a, C2 a) => a
\end{code}
What is meaning of this type? Clearly, \( a \in \texttt{C1} \) and \( a \in \texttt{C2} \) must hold.
Giving familiar notion to this phenomenon we will write following.
\[ \texttt{ex} :: \forall \, a \in \texttt{C1} \cap \texttt{C2} \, . \, a \]
Fragment of emerging structure including standard Prelude classes \texttt{Eq}, \texttt{Ord}, \texttt{Show}, and \texttt{Read}
can be found in figure~\ref{diagram:kinds_star2}.

% \end{example}

\noindent
\begin{figure}
	\centering
	\scalebox{1.0}{
\begin{tikzpicture}[>=latex,line join=bevel,]
  \pgfsetlinewidth{1bp}
%%
\pgfsetcolor{black}
  % Edge: \texttt{Read} -> \texttt{Show} \cap \texttt{Read}
  \pgfsetcolor{gray}
  \draw [->] (232.63bp,167.4bp) .. controls (238.21bp,163.06bp) and (244.82bp,157.92bp)  .. (259.19bp,146.74bp);
  % Edge: \texttt{Eq} \cap \texttt{Read} -> \texttt{Eq} \cap \texttt{Show} \cap \texttt{Read}
  \draw [->] (196.79bp,124.08bp) .. controls (199.91bp,120.62bp) and (203.41bp,116.76bp)  .. (213.58bp,105.52bp);
  % Edge: \texttt{Eq} \cap \texttt{Show} \cap \texttt{Read} -> \texttt{Ord} \cap \texttt{Show} \cap \texttt{Read}
  \draw [->] (195.6bp,82.47bp) .. controls (184.09bp,77.798bp) and (170.67bp,72.351bp)  .. (149.12bp,63.601bp);
  % Edge: \texttt{Ord} \cap \texttt{Show} -> \texttt{Ord} \cap \texttt{Show} \cap \texttt{Read}
  \draw [->] (57.662bp,83.441bp) .. controls (67.824bp,78.707bp) and (79.99bp,73.039bp)  .. (100.41bp,63.523bp);
  % Edge: \texttt{Eq} -> \texttt{Eq} \cap \texttt{Read}
  \draw [->] (120.2bp,169.7bp) .. controls (129.92bp,164.72bp) and (142.63bp,158.22bp)  .. (163.31bp,147.62bp);
  % Edge: \texttt{Eq} \cap \texttt{Show} -> \texttt{Ord} \cap \texttt{Show}
  \draw [->] (84.78bp,124.3bp) .. controls (77.557bp,119.9bp) and (69.215bp,114.83bp)  .. (52.699bp,104.77bp);
  % Edge: \texttt{Eq} \cap \texttt{Read} -> \texttt{Ord} \cap \texttt{Read}
  \draw [->] (168.45bp,124.3bp) .. controls (162bp,120bp) and (154.57bp,115.05bp)  .. (139.16bp,104.77bp);
  % Edge: \texttt{Eq} -> \texttt{Ord}
  \pgfsetcolor{black}
  \draw [->] (87.997bp,168.26bp) .. controls (79.995bp,163.39bp) and (70.108bp,157.37bp)  .. (52.367bp,146.57bp);
  % Edge: \texttt{Ord} \cap \texttt{Show} \cap \texttt{Read} -> \emptyset
  \pgfsetcolor{gray}
  \draw [->,dotted] (123bp,42.228bp) .. controls (123bp,39.648bp) and (123bp,36.8bp)  .. (123bp,23.708bp);
  % Edge: \texttt{Show} \cap \texttt{Read} -> \texttt{Eq} \cap \texttt{Show} \cap \texttt{Read}
  \draw [->] (260.64bp,125.4bp) .. controls (256.02bp,121.45bp) and (250.64bp,116.83bp)  .. (237.57bp,105.63bp);
  % Edge: \kindStar -> \texttt{Read}
  \draw [->] (172.02bp,209.32bp) .. controls (178.56bp,204.96bp) and (186.7bp,199.53bp)  .. (203.01bp,188.66bp);
  % Edge: \texttt{Read} -> \texttt{Eq} \cap \texttt{Read}
  \draw [->] (210.67bp,167.4bp) .. controls (207.86bp,163.82bp) and (204.63bp,159.71bp)  .. (195.14bp,147.63bp);
  % Edge: \texttt{Show} -> \texttt{Eq} \cap \texttt{Show}
  \draw [->] (145.12bp,167.4bp) .. controls (139.7bp,163.26bp) and (133.32bp,158.39bp)  .. (119.23bp,147.63bp);
  % Edge: \kindStar -> \texttt{Show}
  \draw [->] (159bp,208.32bp) .. controls (159bp,205.45bp) and (159bp,202.19bp)  .. (159bp,188.72bp);
  % Edge: \kindStar -> \texttt{Eq}
  \draw [->] (145.97bp,208.52bp) .. controls (140.64bp,204.65bp) and (134.3bp,200.03bp)  .. (120.09bp,189.7bp);
  % Edge: \texttt{Show} -> \texttt{Show} \cap \texttt{Read}
  \draw [->] (180.27bp,169.38bp) .. controls (183.19bp,168.23bp) and (186.17bp,167.07bp)  .. (189bp,166bp) .. controls (203.22bp,160.62bp) and (218.88bp,154.95bp)  .. (242.57bp,146.56bp);
  % Edge: \texttt{Ord} -> \texttt{Ord} \cap \texttt{Show}
  \draw [->] (35bp,125.4bp) .. controls (35bp,122.14bp) and (35bp,118.44bp)  .. (35bp,104.74bp);
  % Edge: \texttt{Eq} \cap \texttt{Show} -> \texttt{Eq} \cap \texttt{Show} \cap \texttt{Read}
  \draw [->] (136.16bp,124.74bp) .. controls (150.05bp,119.88bp) and (166.51bp,114.12bp)  .. (190.92bp,105.58bp);
  % Edge: \texttt{Eq} -> \texttt{Eq} \cap \texttt{Show}
  \draw [->] (104bp,166.08bp) .. controls (104bp,163.44bp) and (104bp,160.57bp)  .. (104bp,147.52bp);
  % Edge: \texttt{Ord} \cap \texttt{Read} -> \texttt{Ord} \cap \texttt{Show} \cap \texttt{Read}
  \draw [->] (123bp,83.228bp) .. controls (123bp,80.286bp) and (123bp,76.996bp)  .. (123bp,63.559bp);
  % Edge: \texttt{Ord} -> \texttt{Ord} \cap \texttt{Read}
  \draw [->] (54.095bp,126.89bp) .. controls (65.051bp,121.66bp) and (79.08bp,114.96bp)  .. (100.84bp,104.58bp);
  % Node: \texttt{Show}
\begin{scope}
  \definecolor{strokecol}{rgb}{0.0,0.0,0.0};
  \pgfsetstrokecolor{strokecol}
  \draw (180bp,189bp) -- (138bp,189bp) -- (138bp,168bp) -- (180bp,168bp) -- cycle;
  \draw (159bp,178bp) node {$\texttt{Show}$};
\end{scope}
  % Node: \texttt{Ord}
\begin{scope}
  \definecolor{strokecol}{rgb}{0.0,0.0,0.0};
  \pgfsetstrokecolor{strokecol}
  \draw (54bp,147bp) -- (16bp,147bp) -- (16bp,126bp) -- (54bp,126bp) -- cycle;
  \draw (35bp,136bp) node {$\texttt{Ord}$};
\end{scope}
  % Node: \texttt{Eq} \cap \texttt{Show} \cap \texttt{Read}
\begin{scope}
  \definecolor{strokecol}{rgb}{0.75,0.75,0.75};
  \pgfsetstrokecolor{strokecol}
  \draw (272bp,106bp) -- (176bp,106bp) -- (176bp,83bp) -- (272bp,83bp) -- cycle;
  \definecolor{strokecol}{rgb}{0.0,0.0,0.0};
  \pgfsetstrokecolor{strokecol}
  \draw (224bp,94bp) node {$\texttt{Eq} \cap \texttt{Show} \cap \texttt{Read}$};
\end{scope}
  % Node: \texttt{Show} \cap \texttt{Read}
\begin{scope}
  \definecolor{strokecol}{rgb}{0.75,0.75,0.75};
  \pgfsetstrokecolor{strokecol}
  \draw (310bp,147bp) -- (236bp,147bp) -- (236bp,126bp) -- (310bp,126bp) -- cycle;
  \definecolor{strokecol}{rgb}{0.0,0.0,0.0};
  \pgfsetstrokecolor{strokecol}
  \draw (273bp,136bp) node {$\texttt{Show} \cap \texttt{Read}$};
\end{scope}
  % Node: \kindStar
\begin{scope}
  \definecolor{strokecol}{rgb}{0.75,0.75,0.75};
  \pgfsetstrokecolor{strokecol}
  \draw (172bp,228bp) -- (146bp,228bp) -- (146bp,209bp) -- (172bp,209bp) -- cycle;
  \definecolor{strokecol}{rgb}{0.0,0.0,0.0};
  \pgfsetstrokecolor{strokecol}
  \draw (159bp,218bp) node {$\kindStar$};
\end{scope}
  % Node: \texttt{Ord} \cap \texttt{Read}
\begin{scope}
  \definecolor{strokecol}{rgb}{0.75,0.75,0.75};
  \pgfsetstrokecolor{strokecol}
  \draw (158bp,105bp) -- (88bp,105bp) -- (88bp,84bp) -- (158bp,84bp) -- cycle;
  \definecolor{strokecol}{rgb}{0.0,0.0,0.0};
  \pgfsetstrokecolor{strokecol}
  \draw (123bp,94bp) node {$\texttt{Ord} \cap \texttt{Read}$};
\end{scope}
  % Node: \texttt{Eq} \cap \texttt{Show}
\begin{scope}
  \definecolor{strokecol}{rgb}{0.75,0.75,0.75};
  \pgfsetstrokecolor{strokecol}
  \draw (136bp,148bp) -- (72bp,148bp) -- (72bp,125bp) -- (136bp,125bp) -- cycle;
  \definecolor{strokecol}{rgb}{0.0,0.0,0.0};
  \pgfsetstrokecolor{strokecol}
  \draw (104bp,136bp) node {$\texttt{Eq} \cap \texttt{Show}$};
\end{scope}
  % Node: \texttt{Eq}
\begin{scope}
  \definecolor{strokecol}{rgb}{0.0,0.0,0.0};
  \pgfsetstrokecolor{strokecol}
  \draw (120bp,190bp) -- (88bp,190bp) -- (88bp,167bp) -- (120bp,167bp) -- cycle;
  \draw (104bp,178bp) node {$\texttt{Eq}$};
\end{scope}
  % Node: \texttt{Eq} \cap \texttt{Read}
\begin{scope}
  \definecolor{strokecol}{rgb}{0.75,0.75,0.75};
  \pgfsetstrokecolor{strokecol}
  \draw (218bp,148bp) -- (154bp,148bp) -- (154bp,125bp) -- (218bp,125bp) -- cycle;
  \definecolor{strokecol}{rgb}{0.0,0.0,0.0};
  \pgfsetstrokecolor{strokecol}
  \draw (186bp,136bp) node {$\texttt{Eq} \cap \texttt{Read}$};
\end{scope}
  % Node: \texttt{Ord} \cap \texttt{Show} \cap \texttt{Read}
\begin{scope}
  \definecolor{strokecol}{rgb}{0.75,0.75,0.75};
  \pgfsetstrokecolor{strokecol}
  \draw (174bp,64bp) -- (72bp,64bp) -- (72bp,43bp) -- (174bp,43bp) -- cycle;
  \definecolor{strokecol}{rgb}{0.0,0.0,0.0};
  \pgfsetstrokecolor{strokecol}
  \draw (123bp,53bp) node {$\texttt{Ord} \cap \texttt{Show} \cap \texttt{Read}$};
\end{scope}
  % Node: \emptyset
\begin{scope}
  \definecolor{strokecol}{rgb}{0.75,0.75,0.75};
  \pgfsetstrokecolor{strokecol}
  \draw (136bp,24bp) -- (110bp,24bp) -- (110bp,1bp) -- (136bp,1bp) -- cycle;
  \definecolor{strokecol}{rgb}{0.0,0.0,0.0};
  \pgfsetstrokecolor{strokecol}
  \draw (123bp,12bp) node {$\emptyset$};
\end{scope}
  % Node: \texttt{Ord} \cap \texttt{Show}
\begin{scope}
  \definecolor{strokecol}{rgb}{0.75,0.75,0.75};
  \pgfsetstrokecolor{strokecol}
  \draw (70bp,105bp) -- (0bp,105bp) -- (0bp,84bp) -- (70bp,84bp) -- cycle;
  \definecolor{strokecol}{rgb}{0.0,0.0,0.0};
  \pgfsetstrokecolor{strokecol}
  \draw (35bp,94bp) node {$\texttt{Ord} \cap \texttt{Show}$};
\end{scope}
  % Node: \texttt{Read}
\begin{scope}
  \definecolor{strokecol}{rgb}{0.0,0.0,0.0};
  \pgfsetstrokecolor{strokecol}
  \draw (240bp,189bp) -- (198bp,189bp) -- (198bp,168bp) -- (240bp,168bp) -- cycle;
  \draw (219bp,178bp) node {$\texttt{Read}$};
\end{scope}
%
\end{tikzpicture}

}
	\caption[Alternative view for some \uv{star} classes]{Alternative view for some \(\kindStar\) classes}
	\label{diagram:kinds_star2}
\end{figure}

We should also notice, that for example class \texttt{Num} forms structure
captured in figure \ref{diagram:num_intersection}. This way it is clear
that not all types with instances for \texttt{Eq} and \texttt{Show}
have to have instance for \texttt{Num}.

Usage of \(\cap\) operator is also one of reasons to include \(\emptyset\)
class.

\noindent
\begin{figure}
	\centering
	\scalebox{1.0}{
\begin{tikzpicture}[>=latex,line join=bevel,]
  \pgfsetlinewidth{1bp}
%%
\pgfsetcolor{black}
  % Edge: \texttt{Eq} -> \texttt{Eq} \cap \texttt{Show}
  \draw [->] (21.206bp,118.43bp) .. controls (24.717bp,110.63bp) and (29.45bp,100.11bp)  .. (37.81bp,81.533bp);
  % Edge: \texttt{Eq} \cap \texttt{Show} -> \texttt{Num}
  \draw [->] (43bp,58.343bp) .. controls (43bp,50.799bp) and (43bp,40.76bp)  .. (43bp,21.837bp);
  % Edge: \texttt{Show} -> \texttt{Eq} \cap \texttt{Show}
  \draw [->] (65.994bp,119.27bp) .. controls (62.352bp,111.47bp) and (57.301bp,100.64bp)  .. (48.538bp,81.868bp);
  % Node: \texttt{Show}
\begin{scope}
  \definecolor{strokecol}{rgb}{0.0,0.0,0.0};
  \pgfsetstrokecolor{strokecol}
  \draw (92bp,141bp) -- (50bp,141bp) -- (50bp,120bp) -- (92bp,120bp) -- cycle;
  \draw (71bp,130bp) node {$\texttt{Show}$};
\end{scope}
  % Node: \texttt{Num}
\begin{scope}
  \definecolor{strokecol}{rgb}{0.0,0.0,0.0};
  \pgfsetstrokecolor{strokecol}
  \draw (62bp,22bp) -- (24bp,22bp) -- (24bp,1bp) -- (62bp,1bp) -- cycle;
  \draw (43bp,11bp) node {$\texttt{Num}$};
\end{scope}
  % Node: \texttt{Eq}
\begin{scope}
  \definecolor{strokecol}{rgb}{0.0,0.0,0.0};
  \pgfsetstrokecolor{strokecol}
  \draw (32bp,142bp) -- (0bp,142bp) -- (0bp,119bp) -- (32bp,119bp) -- cycle;
  \draw (16bp,130bp) node {$\texttt{Eq}$};
\end{scope}
  % Node: \texttt{Eq} \cap \texttt{Show}
\begin{scope}
  \definecolor{strokecol}{rgb}{0.0,0.0,0.0};
  \pgfsetstrokecolor{strokecol}
  \draw (75bp,82bp) -- (11bp,82bp) -- (11bp,59bp) -- (75bp,59bp) -- cycle;
  \draw (43bp,70bp) node {$\texttt{Eq} \cap \texttt{Show}$};
\end{scope}
%
\end{tikzpicture}

}
	\caption[Formation of Num class]{Formation of \texttt{Num} class}
	\label{diagram:num_intersection}
\end{figure}

Another interesting example is pictured on figure
\ref{diagram:kinds_star_star} (please notice that symbol \(\to\)
is overloaded here, representing either type constructor and kind
constructor). It's another place where \(\emptyset\) comes in
handy, as (without any surprise) \(\to\) class constructor
is contravariant in first argument. Arrow between \texttt{Functor}
and \texttt{Monad} is drawn dashed, as there is no such
statement in Prelude, but we all know it's true.

\noindent
\begin{figure}
	\centering
	\scalebox{1.0}{
\begin{tikzpicture}[>=latex,line join=bevel,]
  \pgfsetlinewidth{1bp}
%%
\pgfsetcolor{black}
  % Edge: \kindStar \to \kindStar -> \kindStar \to \texttt{Eq}
  \draw [->] (47.111bp,126.47bp) .. controls (54.966bp,121.82bp) and (64.649bp,116.09bp)  .. (82.411bp,105.59bp);
  % Edge: \emptyset \to \texttt{Eq} -> \emptyset \to \emptyset
  \draw [->] (147.61bp,165.63bp) .. controls (150.36bp,162.41bp) and (153.39bp,158.88bp)  .. (163.09bp,147.57bp);
  % Edge: \texttt{Eq} \to \texttt{Eq} -> \texttt{Eq} \to \emptyset
  \draw [->] (121.5bp,124.3bp) .. controls (128.1bp,120.34bp) and (135.62bp,115.83bp)  .. (151.63bp,106.22bp);
  % Edge: \emptyset \to \texttt{Eq} -> \texttt{Eq} \to \texttt{Eq}
  \draw [->] (126.69bp,165.63bp) .. controls (124.01bp,162.41bp) and (121.06bp,158.88bp)  .. (111.64bp,147.57bp);
  % Edge: \emptyset \to \kindStar -> \emptyset \to \texttt{Eq}
  \draw [->] (111.94bp,208.08bp) .. controls (114.57bp,204.91bp) and (117.49bp,201.41bp)  .. (126.92bp,190.1bp);
  % Edge: \texttt{Eq} \to \texttt{Eq} -> \kindStar \to \texttt{Eq}
  \draw [->] (102bp,124.08bp) .. controls (102bp,121.44bp) and (102bp,118.57bp)  .. (102bp,105.52bp);
  % Edge: \emptyset \to \kindStar -> \texttt{Eq} \to \kindStar
  \draw [->] (92.064bp,208.08bp) .. controls (89.26bp,204.71bp) and (86.135bp,200.96bp)  .. (76.599bp,189.52bp);
  % Edge: \texttt{\texttt{Monad}} -> \kindStar \to \emptyset
  \pgfsetcolor{gray}
  \draw [->,dotted] (55.967bp,42.441bp) .. controls (62.122bp,38.371bp) and (69.321bp,33.61bp)  .. (84.562bp,23.532bp);
  % Edge: \texttt{Eq} \to \kindStar -> \texttt{Eq} \to \texttt{Eq}
  \pgfsetcolor{black}
  \draw [->] (76.936bp,166.08bp) .. controls (79.74bp,162.71bp) and (82.865bp,158.96bp)  .. (92.401bp,147.52bp);
  % Edge: \texttt{Eq} \to \emptyset -> \kindStar \to \emptyset
  \pgfsetcolor{gray}
  \draw [->,dotted] (161.67bp,81.901bp) .. controls (150.38bp,68.669bp) and (132.11bp,47.267bp)  .. (111.92bp,23.618bp);
  % Edge: \texttt{Eq} \to \kindStar -> \kindStar \to \kindStar
  \pgfsetcolor{black}
  \draw [->] (56.78bp,166.08bp) .. controls (53.385bp,162.12bp) and (49.533bp,157.62bp)  .. (39.189bp,145.55bp);
  % Edge: \kindStar \to \kindStar -> \texttt{\texttt{Functor}}
  \draw [->] (30.536bp,126.26bp) .. controls (30.372bp,122.81bp) and (30.18bp,118.78bp)  .. (29.503bp,104.57bp);
  % Edge: \texttt{\texttt{Functor}} -> \texttt{\texttt{Monad}}
  \draw [->,dashed] (31.89bp,83.228bp) .. controls (32.705bp,80.191bp) and (33.619bp,76.783bp)  .. (37.167bp,63.559bp);
  % Edge: \kindStar \to \texttt{Eq} -> \kindStar \to \emptyset
  \pgfsetcolor{gray}
  \draw [->,dotted] (102bp,82.251bp) .. controls (102bp,69.708bp) and (102bp,49.542bp)  .. (102bp,23.727bp);
  % Edge: \emptyset \to \emptyset -> \texttt{Eq} \to \emptyset
  \pgfsetcolor{black}
  \draw [->] (172.72bp,124.08bp) .. controls (172.66bp,121.62bp) and (172.59bp,118.95bp)  .. (172.29bp,106.1bp);
  % Node: \emptyset \to \texttt{Eq}
\begin{scope}
  \definecolor{strokecol}{rgb}{0.0,0.0,0.0};
  \pgfsetstrokecolor{strokecol}
  \draw (163bp,190bp) -- (111bp,190bp) -- (111bp,166bp) -- (163bp,166bp) -- cycle;
  \draw (137bp,178bp) node {$\emptyset \to \texttt{Eq}$};
\end{scope}
  % Node: \texttt{\texttt{Monad}}
\begin{scope}
  \definecolor{strokecol}{rgb}{0.0,0.0,0.0};
  \pgfsetstrokecolor{strokecol}
  \draw (64bp,64bp) -- (16bp,64bp) -- (16bp,43bp) -- (64bp,43bp) -- cycle;
  \draw (40bp,53bp) node {$\texttt{\texttt{Monad}}$};
\end{scope}
  % Node: \kindStar \to \texttt{Eq}
\begin{scope}
  \definecolor{strokecol}{rgb}{0.0,0.0,0.0};
  \pgfsetstrokecolor{strokecol}
  \draw (128bp,106bp) -- (76bp,106bp) -- (76bp,83bp) -- (128bp,83bp) -- cycle;
  \draw (102bp,94bp) node {$\kindStar \to \texttt{Eq}$};
\end{scope}
  % Node: \texttt{Eq} \to \kindStar
\begin{scope}
  \definecolor{strokecol}{rgb}{0.0,0.0,0.0};
  \pgfsetstrokecolor{strokecol}
  \draw (93bp,190bp) -- (41bp,190bp) -- (41bp,167bp) -- (93bp,167bp) -- cycle;
  \draw (67bp,178bp) node {$\texttt{Eq} \to \kindStar$};
\end{scope}
  % Node: \emptyset \to \kindStar
\begin{scope}
  \definecolor{strokecol}{rgb}{0.0,0.0,0.0};
  \pgfsetstrokecolor{strokecol}
  \draw (126bp,232bp) -- (78bp,232bp) -- (78bp,209bp) -- (126bp,209bp) -- cycle;
  \draw (102bp,220bp) node {$\emptyset \to \kindStar$};
\end{scope}
  % Node: \kindStar \to \kindStar
\begin{scope}
  \definecolor{strokecol}{rgb}{0.0,0.0,0.0};
  \pgfsetstrokecolor{strokecol}
  \draw (55bp,146bp) -- (7bp,146bp) -- (7bp,127bp) -- (55bp,127bp) -- cycle;
  \draw (31bp,136bp) node {$\kindStar \to \kindStar$};
\end{scope}
  % Node: \texttt{Eq} \to \texttt{Eq}
\begin{scope}
  \definecolor{strokecol}{rgb}{0.0,0.0,0.0};
  \pgfsetstrokecolor{strokecol}
  \draw (131bp,148bp) -- (73bp,148bp) -- (73bp,125bp) -- (131bp,125bp) -- cycle;
  \draw (102bp,136bp) node {$\texttt{Eq} \to \texttt{Eq}$};
\end{scope}
  % Node: \texttt{\texttt{Functor}}
\begin{scope}
  \definecolor{strokecol}{rgb}{0.0,0.0,0.0};
  \pgfsetstrokecolor{strokecol}
  \draw (58bp,105bp) -- (0bp,105bp) -- (0bp,84bp) -- (58bp,84bp) -- cycle;
  \draw (29bp,94bp) node {$\texttt{\texttt{Functor}}$};
\end{scope}
  % Node: \kindStar \to \emptyset
\begin{scope}
  \definecolor{strokecol}{rgb}{0.75,0.75,0.75};
  \pgfsetstrokecolor{strokecol}
  \draw (126bp,24bp) -- (78bp,24bp) -- (78bp,1bp) -- (126bp,1bp) -- cycle;
  \definecolor{strokecol}{rgb}{0.0,0.0,0.0};
  \pgfsetstrokecolor{strokecol}
  \draw (102bp,12bp) node {$\kindStar \to \emptyset$};
\end{scope}
  % Node: \texttt{Eq} \to \emptyset
\begin{scope}
  \definecolor{strokecol}{rgb}{0.75,0.75,0.75};
  \pgfsetstrokecolor{strokecol}
  \draw (198bp,106bp) -- (146bp,106bp) -- (146bp,82bp) -- (198bp,82bp) -- cycle;
  \definecolor{strokecol}{rgb}{0.0,0.0,0.0};
  \pgfsetstrokecolor{strokecol}
  \draw (172bp,94bp) node {$\texttt{Eq} \to \emptyset$};
\end{scope}
  % Node: \emptyset \to \emptyset
\begin{scope}
  \definecolor{strokecol}{rgb}{0.75,0.75,0.75};
  \pgfsetstrokecolor{strokecol}
  \draw (197bp,148bp) -- (149bp,148bp) -- (149bp,125bp) -- (197bp,125bp) -- cycle;
  \definecolor{strokecol}{rgb}{0.0,0.0,0.0};
  \pgfsetstrokecolor{strokecol}
  \draw (173bp,136bp) node {$\emptyset \to \emptyset$};
\end{scope}
%
\end{tikzpicture}

}
	\caption[Some \uv{emptyset to star} classes]{Some of \(\emptyset \to \kindStar\) classes}
	\label{diagram:kinds_star_star}
\end{figure}

Following example is somehow \uv{problematic}. However, as we will see
later, this is not big deal.

%\begin{example}~
\begin{code}
ex :: (Eq (m a), Monad m) => m a -> a -> Bool
ex x y = x == return y
\end{code}
There are few possibilities how to deal with this. For example
\[ \texttt{ex} :: \forall m \in \texttt{Monad} \cap \emptyset \to \texttt{Eq} \, . \, \forall m \, a \in \texttt{Eq} \, . \, m \, a \to a \to \texttt{Bool} \]
but this would require to make possible quantification not only over variables but over applications too.
% We will rather stick with following alternative:
Alternatively, taking \(\{\_\}\) as a meta-operator mapping type to singleton kind,
it is possible to express desired type by
\[ \texttt{ex} :: \forall a \in \kindStar \, . \, \forall m \in \texttt{Monad} \cap \{ a \} \to \texttt{Eq} \, . \, m \, a \to a \to \texttt{Bool} \]
Both mentioned possibilities embodies some \uv{dependency}, that gives significance to order of quantifications.
We will show how to deal with this later. % XXX : later... multiparam type classes.
%\end{example}

% \begin{example}~
% \begin{code}
% ex :: (Monad m, Eq (m a), Eq a) => m a -> a -> Bool
% ex x y = y == y && x == return y
% \end{code}
% \[ \texttt{ex} :: \forall m \in \texttt{Monad} \cap (\texttt{Eq} \to \texttt{Eq}) \, . \, \forall a \in \texttt{Eq} \, . \, m a \to a \to \texttt{Bool} \]
% \end{example}

This leads to situation, where one th

\section{Mental framework}

In previous section has been pointed out some properties of
Haskell type system that one should take into account when
thinking about it. Following abstract syntax covers what
have been told.

\begin{align*}
	\tau   & ::= \forall \, a {:} \kappa \, . \, \tau ~|~ \sigma \\
	\sigma & ::= a ~|~ c ~|~ \sigma \to \sigma ~|~ \sigma \, \sigma \\
	\kappa & ::= K ~|~ \kindStar ~|~ \kappa \to \kappa ~|~ \kappa \cap \kappa \\
\end{align*}
where
\begin{itemize}
	\item \(a\) stands for type variable,
	\item \(c\) stands for type constant,
	\item \(\tau\) represents types,
	\item \(K\) stands for kind constant,
	\item and \(\kappa\) represents kinds.
\end{itemize}

Key in understanding this fact (noticed by \cite{libor}),
that as classes are sets of type constructors, they are kinds.
Not only that, they form obvious structure~-- intersection-semilattice.

Type class declaration creates new kind constant.
Instances fill selected classes (constants explicitly, others implicitly).
Analogously \texttt{data} creates and fills types.

% XXX: Is it really sort that matters?
Intersection have sense (understand possibility of not-emptiness) only on
kinds with same sort(?) (\(\texttt{Eq} \cap \texttt{Functor}\) even feels wrong).

Inclusion diagrams accompanying some distinguished type classes can be found for
example in \cite{typeclassopedia}.

% XXX: Eventually remove this text and if possible replace
%      with 15 pages of text ;-).
This was expected to generate like 15 pages in thesis \verb~:-/~.

\section{System without type-class infrastructure}

As noticed by many and underlined by actual implementation, type-classes can be removed
from the language with only little \uv{harm}.

Let us have a look at how it can be done. Take the following well known example.
\begin{code}
class Eq a where
    (==), (/=) :: a -> a -> Bool
\end{code}
And definitions that employ this class.
\begin{code}
(==) :: Eq a => a -> a -> Bool
elem :: Eq a => a -> [a] -> Bool
elem = any . (==)
f :: Eq a => a -> [[a]] -> Bool
f = any . elem
\end{code}
This can be expressed in following way:
\begin{code}
type Eq a = (a -> a -> Bool, a -> a -> Bool)
\end{code}
with definitions
\begin{code}
(==) :: Eq a -> a -> a -> Bool
(==) = fst

elem :: Eq a -> a -> [a] -> Bool
-- elem ((==), (/=)) = any . (==)
-- or
elem i = any . (==) i

f :: Eq a -> a -> [[a]] -> Bool
f i = any . elem i
\end{code}

One may naturally wonder, how it would be with non-\(\kindStar\) classes.
Take \texttt{Monad}.
\begin{code}
class Monad m where
	return :: a -> m a
	(>>=) :: m a -> (a -> m b) -> m b
\end{code}
(Function \texttt{fail} is omitted intentionally but can be added with
no problem.)
Most straightforward way how to express it would be
\begin{code}
type Monad m a b = (a -> m a, m a -> (a -> m b) -> m b)
\end{code}
But what have we done? This is not traditional \texttt{Monad}
but multi-parameter (deeper discussed on page
\pageref{extension:multiparam}). This is another example for
multi-parameter type classes being natural. But how to express the
traditional \texttt{Monad}?
\begin{code}
type Monad m = forall a b . (a -> m a, m a -> (a -> m b) -> m b)
\end{code}
But this introduces rank-n or at least rank-2 types (page
\pageref{extension:rankn}).

Naturally arising question is, why would anyone want to do this?
Well, we got rid of all problems that arise with type classes:
monomorphism restriction, overlapping instances and so on.
However, we need to explicitly write what instance should be used.
Also we lost this possibility
to exploit it as way to perform compile-time computation.

\chapter{Extensions}

This chapters goes trough various type extensions showing examples
what can be achieved using them with focus on what they have in common
and where they differ.

\section{Empty data declarations}

% This extension is implemented in \texttt{GHC} and can be enabled via
% \texttt{LANGUAGE} pragma \texttt{EmptyDataDecls}.
The extension described in this paragraph was included in Haskell 2010 standard.

Extension \emph{empty data declarations} allows empty data declarations
(without any data constructors). This is very natural as there is possible
to declare types with arbitrary number of constructors (except 0). In
fact, it is not extension in sense of adding something artificial in
language, it only make syntax more comprehensive. Using this extension can
be empty (except \(\bot\)) data type \texttt{Empty} declared in this way:
\begin{code}
data Empty
\end{code}
However, the same can be achieved without this extension as well. Consider type \texttt{Empty'}
declared with \texttt{newtype}.
\begin{code}
newtype Empty' = E' Empty'
\end{code}
It looks like there actually are values besides \(\bot\), namely \(\texttt{E'}^n \, \bot\)
for every \(n \in \mathbb{N}_0\) and \(\mu x . \texttt{E'} \, x\).
Those are, as shown in lemma~\ref{emptyTypes}, indistinguishable from \(\bot\).

\begin{lemma}[Empty types]
\label{emptyTypes}
If there is value \(v :: \texttt{Empty'}\), then \(v \approx \bot\).
\end{lemma}

\begin{proof}
In \cite{haskell2010} is stated, that \(\texttt{Empty'} \, \bot \approx \bot\).
Simple induction provides us all \(\texttt{E'}^n \, \bot\) cases
and least fixed point case is obvious as well, as \(\bot\) is least element.
\end{proof}

This \uv{extension} is in fact very natural. Definition of \texttt{Empty'}
might give user feeling that it is non-empty, whereas emptiness is
obvious in case of \texttt{Empty}.

Analogous construction can be done with type constructors.

When defining empty type without this extension be sure to use
\texttt{newtype}, not \texttt{data}, as it will be inhabited not only with \(\bot\).
\begin{code}
data EmptyBad = EB EmptyBad
\end{code}

\begin{lemma}
There exists value \(v :: \texttt{EmptyBad}\) such that \(v \not \approx \bot\).
\end{lemma}

\begin{proof}
Let \texttt{emptyBad} be catamorphism over \texttt{EmptyBad}.
\begin{code}
emptyBad :: (a -> a) -> EmptyBad -> a
emptyBad f (EB x) = f (emptyBad x)
\end{code}
Now consider \(\texttt{emptyBad fac (fix EB)}\) and \(\texttt{emptyBad fac} \, \bot\).
They are obviously distinguishable as first one is factorial function but second one is \(\bot\) itself.
\end{proof}

\section{Multi-parameter type classes}
\label{extension:multiparam}

This extension is implemented in \texttt{GHC} and can be enabled via
\texttt{LANGUAGE} pragma \texttt{MultiParamTypeClasses}.

With extension classes are no more simple subsets but relations~-- subsets of Cartesian products.
This requires us to enrich a bit our mental framework.

\begin{align*}
	a^n      & ::= \overbrace{(a,\dots,a)}^n \\
	\tau     & ::= \forall \, a^n {:} \kappa^n \, . \, \tau ~|~ \sigma \\
	\sigma   & ::= a ~|~ c ~|~ \sigma \to \sigma ~|~ \sigma \, \sigma \\
	\kappa^n & ::= K_n ~|~ \overbrace{\kappa \times \dots \times \kappa}^n \\
	\kappa   & ::= K ~|~ \kindStar ~|~ \kappa \to \kappa ~|~ \kappa \cap \kappa \\
\end{align*}

Here
\begin{itemize}
	\item \(a^n\) represents \(n\)-tuples of variables,
	\item \(\kappa^n\) represents a \(n\)-parametric type class (special case for \(n = 1\) is what we had before),
	\item \(K_n\) represents constant \(n\)-parametric type class.
\end{itemize}

% Traditional \texttt{Monad}.
% \begin{code}
% class Functor m => Monad m where
%     return :: a -> m a
%     (>>=) :: m a -> (a -> m b) -> m b
% \end{code}
%
% Multi-parametric \texttt{Monad}.
% \begin{code}
% class Functor m a b => Monad m a b where
%     return :: a -> m a
%     (>>=) :: m a -> (a -> m b) -> m b
% \end{code}

Among others, very nice reason to try to expand the language this way is for example
class for containers. Traditionally, it would look like this.
\begin{code}
class ContainerT c where
    empty  :: c a
    insert :: a -> c a -> c a
\end{code}
but as we have seen earlier (when\dots removing TC infrastructure?) this forces
us to have \texttt{a}-variable quantified universally.

And its multi-parametric version.
\begin{code}
class ContainerM c a where
    empty  :: c a
    insert :: a -> c a -> c a
\end{code}

One would expect instances for lists and sets (among others).  However,
Haskell set implementation allows creating sets only of types that belongs
to \texttt{Ord}, so instance for \texttt{Set} of \texttt{ContainerT}
is not be possible.
Class \texttt{ContainerT} is somewhere under \(\kindStar \to \kindStar\),
but type constructor \texttt{Set} is not even there (it can be found
under \(\texttt{Ord} \to \kindStar\), again contravariance of \(\to\)).

Instance for \texttt{ContainerM} is more or less trivial.
\begin{code}
instance Ord a => ContainerM Set a where
    empty  = Set.empty
    insert = Set.insert
\end{code}

\subsection{Functional dependencies}

Similarly as in relational databases \emph{functional dependencies} refers to
restriction of selected relations (in this case multi-parameter type classes) in the
way that they are functions. (Or at least some of their parameters.)

\subsection{Relaxed rules for instances}

\subsubsection{Relaxed rules for the instance head}

\texttt{TypeSynonymInstances} Not much interesting.

\subsubsection{Relaxed rules for instance contexts}

\begin{itemize}
\item \texttt{FlexibleContexts}
\item \texttt{UndecidableInstances}
\item \texttt{OverlappingInstances} \texttt{IncoherentInstances}
\end{itemize}

\section{Generalised algebraic datatypes}

\section{Extensible kinds}

\section{Rank-n types}
\label{extension:rankn}

\subsection{Rank-2 types}

% \section{Subtyping} % as example

\chapter{Examples}

\section{Apples and oranges}

Problem is simple: we do not want do jumble apples and oranges,
exactly like we were taught in elementary school.

\subsection{Type synonyms}

Simplest way how to \uv{differentiate} fruit under discussion is to define
appropriate type synonyms.
\begin{code}
type Apples  = Integer
type Oranges = Integer
\end{code}
These alternative names can be now used in type signatures to signalise that
particular value represents amount of apples/oranges.

Advantages of this solution include possibility to use all power we have
to manipulate original type (\texttt{Integer} in this case). On the contrary,
safety requirement have not been met, as those type synonyms are equivalent
during type checking. Compiler will have no problem with
code like \texttt{(15 :: Apples) + (27 :: Oranges)}.

\subsection{Type Classes}

Type classes provide another possible solution. We define newtype wrapper
for each fruit type.
\begin{code}
newtype Apples  = Apples  Integer
newtype Oranges = Oranges Integer
\end{code}

Those are fresh new types that only wait for future use.
Addition of two representations of same type of fruit is somehow semantically
always the same, so we want to call it the same name. Overloading of this kind is why
type classes were introduced into the language.
\begin{code}
class Fruit a where
    plus :: a -> a -> a
instance Fruit Apples where
	Apples x `plus` Apples y = Apples (x + y)
instance Fruit Oranges where
	Oranges x `plus` Oranges y = Oranges (x + y)
\end{code}

This way is safety requirement satisfied and function
with same meaning have same names. However, we need to write
very similar definitions for all functions in class
for every new type of fruit over and over. That is clearly
point where unwanted errors can be introduced. So there is still
space for enhancements.


\subsection{Phantom types \& Empty data declarations}

This solution is simple and clean.
\begin{code}
data Apple
data Orange

newtype Count a = Count Integer

plus :: Count a -> Count a -> Count a
Count x `plus` Count y = Count (x + y)
\end{code}
The most interesting thing here is function \texttt{plus}, that itself has a more general type, but
by narrowing it down to more specific one we get a function that allows us
adding up oranges with oranges, apples with apples, but forbids any other combinations
of apples and oranges. Advantage of this solution is that it is polymorphic
and we do not need to separately write code for apples and oranges.
We obeyed abstraction principle very much here.

On the contrary, this can be also seen as drawback, as one can create value
of type \texttt{Count Bool} which we have not intended. However,
one can take this in count by writing:
\begin{code}
data Apple
data Orange

class Fruit a where

instance Fruit Apple
instance Fruit Orange

newtype Count a = Count Integer

plus :: Fruit a => Count a -> Count a -> Count a
Count x `plus` Count y = Count (x + y)
\end{code}

% As every elementary school child notices, from time to time one need to add apples and oranges,
% and result will not be such nonsense:
% \begin{code}
% data Fruit'
% instance Fruit Fruit'
%
% plus' :: (Fruit a, Fruit b) => Count a -> Count b -> Count Fruit'
% Count x `plus'` Count y = Count (x + y)
% \end{code}

\section{Subtyping}

Subtype polymorphism (for more detail see \cite{pierce:2002:types})
is undoubtedly interesting feature. In math it comes natural to use
natural number where some rational or real (or even complex) is expected.
%
One might want to have the ease of use in language
he or she is fond of.

On the other hand, one should not forget, that being
able to do this \uv{on paper} is not without a charge.
Remember those algebra classes showing how to \uv{cook}
various types of numbers. One of steps while building
new in example rationals from integers was
coercion function that did the \uv{conversion}.

Without subtyping one would need to use explicit
coercion functions like \texttt{fromIntegral} or
similar, or even use combination of more than one.

We would however like to move some responsibility
from our shoulders to type system.

\subsection{Multi-parameter type classes}

Key role in subtyping have so called subtyping relation
on types (type \(T\) is subtype of \(S\)).
And as Multi-parameter type classes are relations on types as well,
we can try to use them. Let us have class \texttt{Subtype} providing
\texttt{coer}~-- coercion function.

\begin{code}
class Subtype a b where
    coer :: a -> b
\end{code}

We can create instance for

\begin{code}
instance Subtype Int Float where
    coer = fromIntegral
\end{code}

Journey of subtyping does not end this simple. What we did now
is we gave common name for coercions. But we still need to
think about them, and when should we combine them.
In fact, this might get even harder, as it's not
obvious, where we got by coercion. Therefore we need
some generalisations. For instance for covariant type
constructors, we might construct instances like the
one for \texttt{Maybe}

\begin{code}
instance Subtype a b => Subtype (Maybe a) (Maybe b) where
    coer = maybe Nothing (Just . coer)
\end{code}

Even contravariance can be handled nicely. Instance for functions
even shows that subtyping relation can be defined on both parameters
of \texttt{->} at once.
\begin{code}
instance (Subtype c a,Subtype b d)
    => Subtype (a -> b) (c -> d) where
        coer f = coer . f . coer
\end{code}

We shall not forget about trivial instance. It will be unnecessary
to erase coercions only because of slight change of type.
\begin{code}
instance Subtype a a where
    coer = id
\end{code}

Of course. Subtyping relation is expected to be transitive.
And available syntax provides us with way to express this.
\begin{code}
instance (Subtype a b,Subtype b c) => Subtype a c where
    coer = coer . coer
\end{code}

And one can dive into more exciting stuff as well. For example,
as \texttt{Maybe} is endofunctor it is more than natural to make following instance.
(This one is less general as it can be.)
\begin{code}
instance Subtype a (Maybe a) where
    coer = Just
\end{code}

Another one reveals relation between \texttt{Maybe} and \texttt{[]}.
\begin{code}
instance Subtype (Maybe a) [a] where
    coer = maybe [] (:[])
\end{code}
And by transitivity we have also endofunctor-like coercion for lists as well.

\begin{code}
instance Subtype a (b -> a) where
    coer = const
\end{code}

%\begin{code}
%instance Subtype a () where
%    coer = const ()
%\end{code}

Anyone who have ever met with systems with subtyping (i.e.
\cite{pierce:2002:types}, \cite{luo:99:coercive}) will notice similarities
between instances (especially trivial, transitivity and functions).
Following example is for those who are not familiar with those.

\[
\inferrule[Id]
	{}
	{\iota : A \hookrightarrow A}
\qquad
\inferrule[Trans]
	{\kappa_1 : A \hookrightarrow B, \kappa_2 : B \hookrightarrow C}
	{\kappa_2 \circ \kappa_1 : A \hookrightarrow C}
\]
\[
\inferrule[Arrow]
	{\kappa_1 : C \hookrightarrow A, \kappa_2 : B \hookrightarrow D}
	{\lambda f . \kappa_2 \circ f \circ \kappa_1 : (A \to B) \hookrightarrow (C \to D)}
\]

It would not be right to make impression that this Haskell code
would actually work. Main troubles are caused by overlapping instances,
identity rule and transitivity. It is also circular structure in subtyping
can be problem.

Using multi-parameter type classes we shown how it might be possible to
get rid of need to thing about construction of coercions and we gave
it common name. However, we still need to write it down. One feature
that can make this more smooth is to replace \texttt{\$} operator.
\begin{code}
\{-# LANGUAGE MultiParamTypeClasses #-\}
\{-# LANGUAGE FlexibleInstances #-\}
\{-# LANGUAGE IncoherentInstances #-\}

import Prelude hiding ( ($) )

($) :: Subtype a b => (b -> c) -> a -> c
($) = coer
\end{code}
\begin{comment}
$
\end{comment}
% Comment is hack for syntax highlighting...
And original code could work as well for trivial instance.
(Again notice that we are playing dangerous game here as it uses
flexible instances and incoherent instances.)

% \subsection{Type families and multi-parameter type classes}
%
% As opposed to solution with multi-parameter type classes, we will
% now show, how to be able to \uv{add up} different types of numbers

% \chapter{Unified view}

% \chapter{Own extension proposal}

%\chapter{Z\'aver}
\chapter{Conclusion}

I came to conclusion that violet cows are present mainly in ads.

Oh, and of course:
Someone might think that this work leads to conclusion that Haskell is somehow not good.
That is not true. Big picture. Only good language pushes one to extend it\dots using features
to its limits, to see what can be done, stimulates thinking about things.

\cite{GADT+Kind=Dep}

\clearpage
\phantomsection
\addcontentsline{toc}{chapter}{\bibname}
%\bibliographystyle{apalike}
\bibliographystyle{plain}
\bibliography{bibliography}

\end{document}
